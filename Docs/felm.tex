%
% Dokumentacja do projektu z Semantyk języków programowania
% Rafał Łukaszewski, Filip Pawlak 
% Wrocław 2014
%

\documentclass[11pt,leqno]{article}

\usepackage[polish]{babel}
\usepackage[utf8]{inputenc}
\usepackage[OT4]{fontenc} 
\usepackage{caption}
\usepackage{graphicx}
\usepackage{listings}
\usepackage{parcolumns}
\usepackage{color}
\usepackage{hyperref}
\usepackage{ulem}
\usepackage{amsmath}
\usepackage{bussproofs}

\newcommand{\denote}[1]{\text{$[\![ $#1$ ]\!]$}}

%%%%%%%%%%%%%%%%%%
% Kropka po numerze paragrafu, podparagrafu itp. 

\makeatletter
 \renewcommand\@seccntformat[1]{\csname the#1\endcsname.\quad}
 \renewcommand\numberline[1]{#1.\hskip0.7em}
\makeatother

%%%%%%%%%%%%%%%%%%
% Kropka po numerze tablicy, rysunku i ustawienie czcionki dla etykiety. 

\captionsetup{labelfont=sc,labelsep=period}

%%%%%%%%%%%%%%%%%%
% Inna numeracja wzorów.

\renewcommand{\theequation}{\arabic{section}.\arabic{equation}}

%%%%%%%%%%%%%%%%%%
% Nowe komendy

\title{{\textbf{Reaktywny język funkcyjny \textbf{FElm}}}\\[1ex]
       {\Large Semantyka języków programowania}\\[-1ex]}
\author{Filip Pawlak \and Rafał Łukaszewski}
\date{Wrocław, dnia \today\ r.}

\begin{document}
\thispagestyle{empty}
\maketitle


%%%%%%%%%%%%%%%%%%%%%%%%%%%%%%%%%%%%%%%%%%%%%%%%%%%%%%%%%%%%%%%%%%%%%%%%%%%%%%%%
\section{Temat}
%%%%%%%%%%%%%%%%%%%%%%%%%%%%%%%%%%%%%%%%%%%%%%%%%%%%%%%%%%%%%%%%%%%%%%%%%%%%%%%%

Celem projektu jest zadanie semantyki oraz implementacja interpretera dla reaktywnego języka funkcyjnego będącego minimalnym podzbiorem języka Elm, opisanym jako FElm w pracy: \href{http://people.seas.harvard.edu/~chong/abstracts/CzaplickiC13.html} {Asynchronous Functional Reactive Programming for GUIs}, na której oparty jest projekt.

%%%%%%%%%%%%%%%%%%%%%%%%%%%%%%%%%%%%%%%%%%%%%%%%%%%%%%%%%%%%%%%%%%%%%%%%%%%%%%%%
\section{Podstawowe założenia}
%%%%%%%%%%%%%%%%%%%%%%%%%%%%%%%%%%%%%%%%%%%%%%%%%%%%%%%%%%%%%%%%%%%%%%%%%%%%%%%%

\begin{itemize}
  \item gorliwy
  \item silnie, ale dynamicznie typowany
  \item podstawowe konstrukcje dla sygnałów, tj. lift oraz foldp, ale bez async
  \item podobnie jak w Elmie - brak sygnału sygnałów oraz sygnałów definiowanych rekurencyjnie
  \item typy danych: liczby całkowite, wartości boolowskie i oczywiście sygnały
  \item wejście - zdefiniowany z góry zbiór sygnałów wejściowych
\end{itemize}

%%%%%%%%%%%%%%%%%%%%%%%%%%%%%%%%%%%%%%%%%%%%%%%%%%%%%%%%%%%%%%%%%%%%%%%%%%%%%%%%
\section{Składnia}
%%%%%%%%%%%%%%%%%%%%%%%%%%%%%%%%%%%%%%%%%%%%%%%%%%%%%%%%%%%%%%%%%%%%%%%%%%%%%%%%

Składnia języka jest całkowicie zgodna z \cite[p.~3.1]{CC}, w związku z czym nie podajemy jej definicji ponownie w tym dokumencie.

%%%%%%%%%%%%%%%%%%%%%%%%%%%%%%%%%%%%%%%%%%%%%%%%%%%%%%%%%%%%%%%%%%%%%%%%%%%%%%%%
\section{Semantyka}
%%%%%%%%%%%%%%%%%%%%%%%%%%%%%%%%%%%%%%%%%%%%%%%%%%%%%%%%%%%%%%%%%%%%%%%%%%%%%%%%

Zgodnie z \cite[p.~3.3]{CC} proces interpretacji programu jest podzielony na dwa etapy. Wpierw wykonywana jest redukcja wszystkich konstrukcji funkcyjnych do termu w języku pośrednim. W tej postaci  pozostają już tylko konstrukcje sygnałów i korzystamy z niej w procesie rozwiązywania zależności pomiędzy konstrukcjami sygnałowymi. Wzorując się na intuicjach zawartych w \cite[p.~3.3.2]{CC} explicite tworzymy graf sygnału w trakcie etapu drugiego, z którego to grafu ostatecznie korzystamy podczas przetwarzania nadchodzących zdarzeń.


Sygnały definiujemy jako strumienie dyskretnych wartości, oraz przetwarzamy je synchronicznie, tj. kiedy w sygnale wejściowym pojawi się nowa wartość musimy rozpropagować ją w całym grafie zanim będziemy mogli obsłużyć nowe zdarzenie. Przyjmujemy, że zbiór sygnałów wejściowych jest stały i zdefiniowany z góry. Podobnie jak w \nocite{CC} (mimo, że nie jest to dosłownie wyszczególnione), czas również traktujemy jak zwykły sygnał, tj. abstrahujemy od pojęcia czasu podczas definiowania semantyki.

\subsection{Ewaluacja funkcyjna}

Podczas pierwszego etapu interpretacji korzystamy bezpośrednio z semantyki operacyjnej opisanej w \cite[p.~3.3.1]{CC}, z regułami zdefiniowanymi w \cite[fig.~6]{CC}. 

\subsection{Budowa grafu}
Ten etap ewaluacji opisujemy w formacie semantyki denotacyjnej. Wierzchołki w grafie reprezentujemy następująco:
\begin{gather*}
Vertex : N \times Expr \times \left\{ {LiftV \times Expr, FoldpV \times Expr, InputV}\right\} 
\end{gather*}

Pierwszy element krotki to numer wierzchołka, drugi to jego wartość, zaś trzeci określa jego typ (wierzchołki typu LiftV i FoldpV dodatkowo przechowują funkcję przekazaną jako argument w wywołaniu odpowiednio \texttt{lift$_{n}$} i \texttt{foldp}). Pomijamy definicję funkcji value : Vertex $\rightarrow$ Expr, która dla danego wierzchołka zwraca jego wartość. Wymagamy by krawędzie pomiędzy dwoma wierzchołkami były numerowane (z uwagi na to, że kolejność argumentów sygnałowych jest istotna):
\begin{gather*}
Edge : Vertex \times Vertex \times N
\end{gather*}

Pomijamy reprezentację grafu, jednak żądamy by istniała funkcja Next : Graph $\rightarrow$ N, która dla danego grafu podaje kolejny dostępny numer wierzchołka. Poprzez N oznaczamy funkcję typu Expr $\rightarrow$ Expr, będącą funkcją semantyczną zadaną przez reguły ewaluacji funkcyjnej. Definiujemy również środowisko:
\begin{gather*}
Env : Var \rightarrow Vertex  
\end{gather*}

wraz z operacją update typu Var $\rightarrow$ Vertex $\rightarrow$ Env $\rightarrow$ Env. Definiujemy funkcję semantyczną D:

\begin {align*}
D : Expr \rightarrow Env \rightarrow Graph \rightarrow Vertex \times Graph
\end {align*}
\begin {align*}
D\denote{x}\:env\:g &= \langle env\:x,\:g \rangle
\end {align*}
\begin {align*}
D\denote{\texttt{lift$_{n}$} f s$_{1}$ ... s$_{n}$} &= \langle v,\:g' \rangle
\end {align*}
\begin {align*}
\text{gdzie} \: \langle v_{1},\:g_{1} \rangle &= D\denote{s$_{1}$}\:env\:g \\
\langle v_{i},\:g_{i} \rangle &= D\denote{s$_{i}$}\:env\:g_{i-1} \quad \text{dla}\:2 \leq i \leq n \\
defaultV &= N\denote{f (value v$_{1}$) ... (value v$_{n}$)} \\
v &= \langle Next(g_{n}),\:defaultV,\:\langle LiftV,\:f \rangle \rangle \\
V(g') &= V(g_{n}) \cup \left\{ {v}\right\} \\
E(g') &= E(g_{n}) \cup \{\langle v_{i},\:v,\:i \rangle : 1 \leq i \leq n\}
\end {align*}
\begin {align*}
D\denote{\texttt{foldp} f d s} = \langle v,\:g' \rangle
\end {align*}
\begin {align*}
\text{gdzie} \: \langle v_{s},\:g_{s} \rangle &= D\denote{s}\:env\:g \\
v &= \langle Next(g_{s}),\:d,\:\langle Foldp,\:f \rangle \rangle \\
V(g') &= V(g_{s}) \cup \left\{ {v}\right\} \\
E(g') &= E(g_{s}) \cup \{\langle v_{s},\:v,\:1 \rangle \}
\end {align*}
\begin {align*}
D\denote{\texttt{let} l = s \texttt{in} r} = \langle v_{r},\:g_{r} \rangle
\end {align*}
\begin {align*}
\text{gdzie} \: \langle v_{s},\:g_{s} \rangle &= D\denote{s}\:env\:g \\
env' &= update\:l\:v_{s}\:env \\
\langle v_{r},\:g_{r} \rangle &= D\denote{r}\:env'\:g_{s}
\end{align*}

\subsection{Inne podejście}

Powyższa semantyka wiernie bazuje na pracy \cite{CC} i klarownie oddziela od siebie ewaluację elementów funkcyjnych i sygnałowych. Wadą takiego podejścia jest nieoczekiwane zatrzymywanie się ewaluacji dla niektórych wyrażeń, jak np.:$\newline$ $\newline$
\texttt{let y = (let x = Window.width in $\backslash$z -> x) $\newline$ in y ()} $\newline$ $\newline$
Jednym z możliwych rozwiązań tej sytuacji jest połączenie obu etapów ewaluacji w jeden i budowanie grafu równolegle z ewaluacją elementów funkcyjnych. Nową relację redukcji oznaczamy w ten sam sposób, lecz teraz jest ona określona na zbiorze Expr $\times$ Graph. Do składni abstrakcyjnej dodajemy pomocnicze wyrażenia \texttt{signal} i (i $\in$ N) reprezentujące sygnały (wierzchołki w grafie). Dodajemy je również do kategorii wartości. Konteksty ewaluacyjne definiujemy niemal identycznie, jednak z definicji kontekstu usuwamy produkcję której prawą stroną jest \texttt{let} x = s \texttt{in} E (zatem nie ewaluujemy ciała konstrukcji \texttt{let}). Definiujemy regułę CONTEXT dla nowej relacji:

\begin{prooftree}
\AxiomC{$\langle$e, g$\rangle$ $\rightarrow$ $\langle$e$'$, g$'$$\rangle$}
\RightLabel{\quad CONTEXT}
\UnaryInfC{$\langle$E[e], g$\rangle$ $\rightarrow$ $\langle$E[e$'$], g$\rangle$}
\end{prooftree}

Do naszej semantyki włączamy reguły APPLICATION, OP, COND-TRUE, COND-FALSE i REDUCE (\cite[p.~3.3.1]{CC}) wraz z poniższą regułą umożliwiającą ich stosowanie:

\begin{prooftree}
\AxiomC{e $\rightarrow$ e$'$}
\UnaryInfC{$\langle$e, g$\rangle$ $\rightarrow$ $\langle$e$'$, g$\rangle$}
\end{prooftree}

Kolejne dwie reguły budują fragment grafu odpowiadający rozpatrywanej konstrukcji sygnałowej oraz zastępują ją wyrażeniem \texttt{signal} i, gdzie i to numer wierzchołka odpowiadającego wynikowemu sygnałowi:

\begin{prooftree}
\AxiomC{}
\RightLabel{\quad LIFT}
\UnaryInfC{$\langle$\texttt{lift$_{n}$} val (\texttt{signal} i$_{1}$) ... (\texttt{signal} i$_{n}$), g$\rangle$ $\rightarrow$ $\langle$\texttt{signal} i, g$'$$\rangle$}
\end{prooftree}
\begin {align*}
\text{gdzie} \quad \quad \: defaultV &= N\denote{val (value v$_{i_{1}}$) ... (value v$_{i_{n}}$)} \\
i &= Next(g) \\
v &= \langle i,\:defaultV,\:\langle LiftV,\:val \rangle \rangle \\
V(g') &= V(g) \cup \left\{ {v}\right\} \\
E(g') &= E(g) \cup \{\langle v_{i_{j}},\:v,\:j \rangle : 1 \leq j \leq n\}
\end{align*}


\begin{prooftree}
\AxiomC{}
\RightLabel{\quad FOLDP}
\UnaryInfC{$\langle$\texttt{foldp} val$_{1}$ val$_{2}$ (\texttt{signal} i), g$\rangle$ $\rightarrow$ $\langle$\texttt{signal} i$'$, g$'$$\rangle$}
\end{prooftree}
\begin {align*}
\text{gdzie} \quad \quad \quad \quad \: v &= \langle Next(g),\:val_{2},\:\langle FoldpV,\:val_{1} \rangle \rangle \\
V(g') &= V(g) \cup \left\{ {v}\right\} \\
E(g') &= E(g) \cup \{\langle v_{i},\:v,\:1 \rangle \}
\end{align*}
\subsection{Propagacja zdarzeń w grafie}

Aby reprezentować pojawienie się nowej wartości w sygnale lub jej brak dodajemy do krawędzi grafu dodatkową informację typu:
$$ type \: Event \: \alpha = NoChange \: \alpha \: | \: Change \: \alpha $$
Pojawienie się nowej wartości w wierzchołku będziemy propagować przez ustawienie na wychodzących z niego krawędziach etykiety $Change\:v$ (gdzie $v$ oznacza nową wartość znajdującą się w wierzchołku), natomiast gdy nic nowego nie pojawi się w wierzchołku oznaczymy jego krawędzie etykietami $NoChange\:v$ (w tym przypadku $v$ to ostatnia wartość jaka pojawiła się w danym sygnale). W ten sposób unikniemy ponownego obliczania wartości w wierzchołkach, kiedy nie jest to konieczne.

Kiedy posiadamy już poprawnie zbudowany graf propagację nowych wartości pojawiających się w sygnale wejściowym przeprowadzamy w trzech krokach:
\begin{enumerate}
\item oznaczenie krawędzi wychodzących wierzchołka danego sygnału wejściowego etykietą $Change\:v$
\item oznaczenie krawędzi wychodzących reszty wierzchołków sygnałów wejściowych etykietą $NoChange\:v$ 
\item odpowiednie przetworzenie każdego wierzchołka (nie reprezentującego sygnału wejściowego) w kolejności wyznaczonej przez porządek topologiczny grafu zgodnie z regułami przedstawionymi poniżej:
\end{enumerate}

Dla grafu $g$ i wierzchołka $v = \langle d, \langle FoldpV,f \rangle \rangle$:
\begin{itemize}
\item jeśli $ \langle v_s , v, 1, Change\:val \rangle \in E(g) $, to w wynikowym grafie $g'$ mamy
\begin {align*}
V(g') &= V(g) \setminus \{ v \} \cup \{ nV \} \\
E(g') &= \begin{aligned} E(g) &\setminus \{ \langle v, \_, _, \_ \rangle : \langle v, \_ , _, \_ \rangle \in E(g) \} \\ 
&\cup \{ \langle v, v_t, k, Change\: nVal \rangle : \langle v, v_t , k, \_ \rangle \in E(g) \} \end{aligned} \\
\text{gdzie} \: nVal &= N [\![f \: val \: d]\!] \\
nV &= \langle nVal , \langle FoldpV,f \rangle \rangle
\end{align*}
\item w innym przypadku:
\begin{align*}
E(g') = E(g) &\setminus \{ \langle v, \_, _, \_ \rangle : \langle v, \_ , _, \_ \rangle \in E(g) \} \\ 
&\cup \{ \langle v, v_t, k, NoChange\: d \rangle : \langle v, v_t , k, \_ \rangle \in E(g) \}
\end{align*}
\end{itemize}

Dla grafu $g$ i wierzchołka $v = \langle d, \langle LiftV_n,f \rangle \rangle$:
\begin{itemize}
\item jeśli $ \exists i . \langle v_s , v, i, Change \: \_ \rangle \in E(g) $, to w wynikowym grafie $g'$ mamy
\begin {align*}
V(g') &= V(g) \setminus \{ v \} \cup \{ nV \} \\
E(g') &= \begin{aligned} E(g) &\setminus \{ \langle v, \_, _, \_ \rangle : \langle v, \_ , _, \_ \rangle \in E(g) \} \\ 
&\cup \{ \langle v, v_t, k, Change\: nVal \rangle : \langle v, v_t , k, \_ \rangle \in E(g) \} \end{aligned} \\
\text{gdzie} \:
nV &= \langle nVal , \langle Lift_n,f \rangle \rangle \\
nVal &= N [\![f \: vals(1) \ldots vals(n) \: d]\!] \\
vals &= \{ \texttt{bodyOf} \: e : \langle \_ , v, i, e \rangle \in E(g), 1 \leq i \leq n \} \\
\texttt{bodyOf} \: e \: &= \texttt{case} \: e \: \texttt{of} \: NoChange \: a \rightarrow a \: | \: Change \: a \rightarrow a
\end{align*}
\item w przeciwnym razie:
\begin{align*}
E(g') = E(g) &\setminus \{ \langle v, \_, \_, \_ \rangle : \langle v, \_ , \_, \_ \rangle \in E(g) \} \\ 
&\cup \{ \langle v, v_t, k, NoChange\: d \rangle : \langle v, v_t , k, \_ \rangle \in E(g) \}
\end{align*}
\end{itemize}

Obsługa strumienia zdarzeń to po prostu sekwencyjne stosowanie powyższej strategii.

%%%%%%%%%%%%%%%%%%%%%%%%%%%%%%%%%%%%%%%%%%%%%%%%%%%%%%%%%%%%%%%%%%%%%%%%%%%%%%%%
\section{Uwagi dotyczące implementacji}
%%%%%%%%%%%%%%%%%%%%%%%%%%%%%%%%%%%%%%%%%%%%%%%%%%%%%%%%%%%%%%%%%%%%%%%%%%%%%%%%

Zaimplementowane zostały obie wersje semantyki budowy grafu, o których mowa w pkt. 4.2 i 4.3. Pierwsza jest zastosowana w funkcji \\ \texttt{Signal.buildGraph} (i \texttt{Functional.normalize}), natomiast druga w funkcji \texttt{Signal.sigNormalize}.

Zbiór sygnałów wejściowych jest w implementacji wprowadzany w postaci swego rodzaju preludium, tj. obsługa właściwego programu zaczyna się na wcześniej utworzonym środowisku oraz grafie, w którym już znajdują się dobrze zdefiniowane wierzchołki sygnałów wejściowych.

%%%%%%%%%%%%%%%%%%%%%%%%%%%%%%%%%%%%%%%%%%%%%%%%%%%%%%%%%%%%%%%%%%%%%%%%%%%%%%%
%%%%%%%%%%%%%%%%%%%%%%%%%%%%%%%%%%%%%%%%%%%%%%%%%%%%%%%%%%%%%%%%%%%%%%%%%%%%%%%%
%% Bibliografia
%%%%%%%%%%%%%%%%%%%%%%%%%%%%%%%%%%%%%%%%%%%%%%%%%%%%%%%%%%%%%%%%%%%%%%%%%%%%%%%
%%%%%%%%%%%%%%%%%%%%%%%%%%%%%%%%%%%%%%%%%%%%%%%%%%%%%%%%%%%%%%%%%%%%%%%%%%%%%%%%
%\newpage
\thispagestyle{empty}
\begin{thebibliography}{99}

\bibitem{CC}   E.~Czaplicki, S.~Chong, \textit{Asynchronous Functional Reactive Programming for GUIs},
              PLDI’13, 2013.
	  
\end{thebibliography}

\end{document}
