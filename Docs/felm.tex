%
% Dokumentacja do projektu z Semantyk języków programowania
% Rafał Łukaszewski, Filip Pawlak 
% Wrocław 2014
%

\documentclass[11pt,leqno]{article}

\usepackage[polish]{babel}
\usepackage[utf8]{inputenc}
\usepackage[OT4]{fontenc} 
\usepackage{caption}
\usepackage{graphicx}
\usepackage{listings}
\usepackage{parcolumns}
\usepackage{color}
\usepackage{hyperref}
\usepackage{ulem}
\usepackage{amsmath}
\usepackage{bussproofs}

%%%%%%%%%%%%%%%%%%
% Kropka po numerze paragrafu, podparagrafu itp. 

\makeatletter
 \renewcommand\@seccntformat[1]{\csname the#1\endcsname.\quad}
 \renewcommand\numberline[1]{#1.\hskip0.7em}
\makeatother

%%%%%%%%%%%%%%%%%%
% Kropka po numerze tablicy, rysunku i ustawienie czcionki dla etykiety. 

\captionsetup{labelfont=sc,labelsep=period}

%%%%%%%%%%%%%%%%%%
% Inna numeracja wzorów.

\renewcommand{\theequation}{\arabic{section}.\arabic{equation}}

%%%%%%%%%%%%%%%%%%
% Nowe komendy

\title{{\textbf{Reaktywny język funkcyjny \textbf{FElm}}}\\[1ex]
       {\Large Semantyka języków programowania}\\[-1ex]}
\author{Filip Pawlak \and Rafał Łukaszewski}
\date{Wrocław, dnia \today\ r.}

\begin{document}
\thispagestyle{empty}
\maketitle


%%%%%%%%%%%%%%%%%%%%%%%%%%%%%%%%%%%%%%%%%%%%%%%%%%%%%%%%%%%%%%%%%%%%%%%%%%%%%%%%
\section{Temat}
%%%%%%%%%%%%%%%%%%%%%%%%%%%%%%%%%%%%%%%%%%%%%%%%%%%%%%%%%%%%%%%%%%%%%%%%%%%%%%%%

Celem projektu jest zadanie semantyki oraz implementacja interpretera dla reaktywnego języka funkcyjnego będącego minimalnym podzbiorem języka Elm, opisanym jako FElm w pracy: \href{http://people.seas.harvard.edu/~chong/abstracts/CzaplickiC13.html} {Asynchronous Functional Reactive Programming for GUIs}, na której oparty jest projekt.

%%%%%%%%%%%%%%%%%%%%%%%%%%%%%%%%%%%%%%%%%%%%%%%%%%%%%%%%%%%%%%%%%%%%%%%%%%%%%%%%
\section{Podstawowe założenia}
%%%%%%%%%%%%%%%%%%%%%%%%%%%%%%%%%%%%%%%%%%%%%%%%%%%%%%%%%%%%%%%%%%%%%%%%%%%%%%%%

\begin{itemize}
  \item gorliwy
  \item silnie, ale dynamicznie typowany
  \item podstawowe konstrukcje dla sygnałów, tj. lift oraz foldp, ale bez async
  \item podobnie jak w Elmie - brak sygnału sygnałów oraz sygnałów definiowanych rekurencyjnie
  \item typy danych: liczby całkowite, wartości boolowskie i oczywiście sygnały
  \item wejście - zdefiniowany z góry zbiór sygnałów wejściowych
\end{itemize}

%%%%%%%%%%%%%%%%%%%%%%%%%%%%%%%%%%%%%%%%%%%%%%%%%%%%%%%%%%%%%%%%%%%%%%%%%%%%%%%%
\section{Składnia}
%%%%%%%%%%%%%%%%%%%%%%%%%%%%%%%%%%%%%%%%%%%%%%%%%%%%%%%%%%%%%%%%%%%%%%%%%%%%%%%%

Składnia języka jest całkowicie zgodna z \cite[p.~3.1]{CC}, w związku z czym nie podajemy jej definicji ponownie w tym dokumencie.

%%%%%%%%%%%%%%%%%%%%%%%%%%%%%%%%%%%%%%%%%%%%%%%%%%%%%%%%%%%%%%%%%%%%%%%%%%%%%%%%
\section{Semantyka}
%%%%%%%%%%%%%%%%%%%%%%%%%%%%%%%%%%%%%%%%%%%%%%%%%%%%%%%%%%%%%%%%%%%%%%%%%%%%%%%%

\subsection{Ewaluacja funkcyjna}

\cite[p.~3.3.1]{CC}
\cite[fig. 6]{CC}

\subsection{Budowa grafu}

$[\![$ semantyka denotacyjna $]\!]$

\begin{prooftree}
\AxiomC{A}
\AxiomC{B}
\AxiomC{C}
\TrinaryInfC{D}
\end{prooftree}

\begin{prooftree}
\AxiomC{A}
\AxiomC{B}
\AxiomC{C}
\BinaryInfC{D}
\BinaryInfC{E}
\end{prooftree}

\begin{prooftree}
\AxiomC{$P$}
\AxiomC{}
\RightLabel{\scriptsize(1)}
\UnaryInfC{$\neg P$}
\BinaryInfC{$\bot$}
\RightLabel{\scriptsize(1)}
\UnaryInfC{$\neg\neg P$}
\end{prooftree}

\subsection{Inne podejście}

semantyka operacyjna

\subsection{Propagacja zdarzeń w grafie}

%%%%%%%%%%%%%%%%%%%%%%%%%%%%%%%%%%%%%%%%%%%%%%%%%%%%%%%%%%%%%%%%%%%%%%%%%%%%%%%%
\section{Uwagi dot. implementacji}
%%%%%%%%%%%%%%%%%%%%%%%%%%%%%%%%%%%%%%%%%%%%%%%%%%%%%%%%%%%%%%%%%%%%%%%%%%%%%%%%

%%%%%%%%%%%%%%%%%%%%%%%%%%%%%%%%%%%%%%%%%%%%%%%%%%%%%%%%%%%%%%%%%%%%%%%%%%%%%%%
%%%%%%%%%%%%%%%%%%%%%%%%%%%%%%%%%%%%%%%%%%%%%%%%%%%%%%%%%%%%%%%%%%%%%%%%%%%%%%%%
%% Bibliografia
%%%%%%%%%%%%%%%%%%%%%%%%%%%%%%%%%%%%%%%%%%%%%%%%%%%%%%%%%%%%%%%%%%%%%%%%%%%%%%%
%%%%%%%%%%%%%%%%%%%%%%%%%%%%%%%%%%%%%%%%%%%%%%%%%%%%%%%%%%%%%%%%%%%%%%%%%%%%%%%%
%\newpage
\thispagestyle{empty}
\begin{thebibliography}{99}

\bibitem{CC}   E.~Czaplicki, S.~Chong, \textit{Asynchronous Functional Reactive Programming for GUIs},
              PLDI’13, 2013.
	  
\end{thebibliography}

\end{document}
